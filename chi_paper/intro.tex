\section{Introduction}\label{sec:intro}

Video games have become one of the most popular entertainments for children, teenagers, even young adults. Consoles such as Nintendo and Sony PlayStation are considered as the pioneers that popularize video games in our society. However, most video games only require the users to press the buttons on the controller in which the results will be displayed on the built-in screen or TV. There is no other interface between the users and the system besides pressing the button in the handheld controller. In 2006, Nintendo released Wii, a console that uses a remote controller which detects users’ movements in 3-D. This console introduced a new and interactive way of playing video games because it requires the user to be physically moving the controller instead of only pressing the buttons.

In this paper, we present \tube, an interactive and tangible user interface system that explores a unique kind of interaction with the user, \ie through breathing control. We will discuss the features and possibilities of \tube becoming a new system to be used on interactive applications and games. There has been a number of existing works which examine the same interaction, but none has been aimed for a general entertainment system.

\textbf{Tooka~\cite{tooka}:} This is a tube which acts like a musical instrument. Two players can blow from the opposite ends of the tube and produce sound collaboratively by controlling their breathings using their tongues and lungs and by pressing the buttons. The main difference of this tube and our \tube is in the output produced. While the Tooka focusesmore on producing sound like a musical instrument, our tube is simply a device used for interaction between the users and our software which focuses primarily on gaming. Our tube is also capable for more different user interaction as it detects rotation and translation while the Tooka seems to support only pressure modulation and button inputs.

\textbf{BLUI~\cite{blui}:} BLUI (Low-cost Localized Blowable User Interface) is a hands-free interaction between user and computer by blowing the computer screen to control the interactive application. Both our systems seem to be able to support similar application; however BLUI supports blowing directly to the computer screen while \tube uses a tube as a device for interaction.

\textbf{The Pipe~\cite{thepipe}:} This is a music input device using breath pressure as control input. Pipe that is used in this projects uses similar sensors that are used in our tube such as accelerometer and force sensing sensor. The main difference is the output; The Pipe focuses on building a musical instrument while \tube acts as a device for users to control out application.
