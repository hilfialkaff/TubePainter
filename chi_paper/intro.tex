\section{Introduction}\label{sec:intro}

There has been a few works that explores user interface system that utilizes breathing.

\textbf{Tooka~\cite{tooka}:} This is a tube which acts like a musical instrument. Each player can blow from the opposite end of the tube and produce the sound collaboratively using their tongues and lungs, and by pressing the buttons. The main difference of this tube and our \tube is in the output produced. While the Tooka focused more on producing sound as a musical instrument, our tube is simply a device used for interaction between users and our software which focus primarily on gaming. Our tube is also capable for more different user interaction as it detects rotation and 2-D movement while the Tooka seems to support only pressure modulation and button inputs.

\textbf{BLUI~\cite{blui}:} BLUI (Low-cost Localized Blowable User Interface) is a hands-free interaction between user and computer by blowing the computer screen to control the interactive application. Both our systems seem to be able to support similar application; however BLUI supports blowing directly to the computer screen while \tube uses a tube as a device for interaction.

\textbf{The Pipe~\cite{thepipe}:} This is a music input device using breath pressure as control input. Pipe that is used in this projects uses similar sensors that are used in our tube such as accelerometer and force sensing sensor. The main difference is the output; The Pipe focuses on building a musical instrument while \tube acts as a device for users to control out application.

\TODO
- Traditional entertainment software: PS, nintendo and newer game console: wii

Video games have become one of the most popular entertainments for most children and teenager, even young adults. Console such as Nintendo and Sony PlayStation were considered as the pioneers that popularize video games in our society. However, most video games only require the users to press the buttons and the result will be displayed on the screen, built-in screen or TV. There is no other interface between the users and the system besides pressing the button in the handheld. On 2006, Nintendo released Wii, a console that uses a remote controller to detect users’ movement in 3-D. This console introduced a new and interactive way of playing video games because it involves user moving the remote instead of only pressing the buttons. 

In this paper, we present \tube, an interactive and tangible system that employs breathing control. We will discuss the features and possibilities of \tube becoming a new system to use an interactive applications and games.
