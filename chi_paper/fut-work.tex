\section{Discussions and Future Work}\label{sec:fut-work}
In the future, we would like to explore on how well \tube works in a collaborative setting. For instance, in the shooting game that we developed, we could extend the game so that two people will compete to get a higher score or collaborate to shoot down a number of balloons under limited time. We believe that our \tube will be much more interactive if this is implemented.

Following up our previous point, making \tube wireless is essential to maximize the user experience. In our current prototype, our \tube is wired to the laptop. In this case, the users will be constrained at how long the wire connecting the tube to the computer is and could not move as freely when using the applications. Under group setting, this problem is exacerbated since multiple users could now collided with each other due to space constraints and this will definitely detriment the user experience.

Instead of utilizing computer screen as the output of our \tube, we believe that it will be best if it is displayed on a standalone screen such as a television screen so that the \tube system will feel more natural to the user and not just "another computer application".

Instead of utilizing computer screen as the output of our \tube, we believe that it will be best if it is displayed on an interactive output such as a table tops so that the \tube system will have more natural user interaction and it will enhance the interactiveness of the system. This way the \tube system will feel more natural to the user and not just "another computer application".

Last but not least, in the hardware side, we also need to smoothen the reading that we get from the accelerometers and gyroscopes in order to maximize user experience.

