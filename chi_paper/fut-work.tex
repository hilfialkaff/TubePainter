\section{Discussions and Future Work}\label{sec:fut-work}

In the future, we would like to explore how well \tube works in a collaborative setting. For instance, for the balloon popping game that we developed, we could extend the game so that two players may compete to get the highest score or collaborate to shoot down a number of balloons under a limited time. This will definitely ameliorate the interactivity and persistence of our tube.

Following up to our previous point, a wireless \tube is also essential to maximize the user experience. Our current \tube prototype is still wired to the laptop. During the tryout, we distinctly observe how our testers were still constrained the length of the wire extending from the tube to the computer and therefore, could not move freely when using the applications. In a collaborative setting, this problem will be exacerbated since each of the users might impede the others' progress in the game due to space constraints which will detriment the user experience.

Finally, instead of merely utilizing computer screen as the output device of \tube, we believe that \tube output is best displayed on an unconventional output system such as a tabletop since that will enhance the interactiveness of our \tube. This way our \tube system will not be regarded by the users as merely "another computer application".
