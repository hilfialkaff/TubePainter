\section{Background}\label{sec:background}

The Tooka~\cite{tooka} is a tube intended for musical instrument that contains pressure sensor and buttons for each player. Each player can blow from the opposite end of the tubes and produce the sound collaboratively using their tongues and lungs and pressing the buttons. The main difference of this tube and our Tangible Tube is in the output produced. While the Tooka focused more on producing sound as a musical instrument, our tube is simply a device used for interaction between users and our software which focus primarily on gaming. Our tube is also capable for more different user interaction as it detects rotation and 2-D movement while the Tooka seems to support only pressure modulation and button inputs.

BLUI: Low-cost Localized Blowable User Interface~\cite{blui} is a hands-free interaction between user and computer by blowing the computer screen to control the interactive application. Both our systems seem to be able to support similar application; however BLUI supports blowing directly to the computer screen while Tangible Tube uses a tube as a device for interaction.

The Pipe~\cite{thepipe} is a music input device using breath pressure as control input. Pipe that is used in this projects uses similar sensors that are used in our tube such as accelerometer and force sensing sensor. The main difference is the output; The Pipe~\cite{thepipe} focuses on building a musical instrument while Tangible Tube acts as a device for users to control out application.


\TODO
- Related works that utilize breathing \newline
- Research projects for entertainment that uses tangible UI
